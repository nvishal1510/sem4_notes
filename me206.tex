\documentclass{article}
\title{ME 206 Manufacturing Processes I (S1)}
\author{Vishal Neeli}

\usepackage[a4paper, total={6in, 11in}]{geometry}
\usepackage{textcomp}
\usepackage{hyperref}
\usepackage{amsmath}
\usepackage{xcolor}
\usepackage{grffile}
\usepackage{physics}
\hypersetup{
	colorlinks=true,
	urlcolor=blue,
	linkcolor=cyan,
	filecolor=red
}
\usepackage{amsfonts}

% FOR CODE
% \usepackage{listings}
% \usepackage{color}

% \definecolor{dkgreen}{rgb}{0,0.6,0}
% \definecolor{gray}{rgb}{0.5,0.5,0.5}
% \definecolor{mauve}{rgb}{0.58,0,0.82}

% \lstset{frame=tb,
%   language=Java,
%   aboveskip=3mm,
%   belowskip=3mm,
%   showstringspaces=false,
%   columns=flexible,
%   basicstyle={\small\ttfamily},
%   numbers=none,
%   numberstyle=\tiny\color{gray},
%   keywordstyle=\color{blue},
%   commentstyle=\color{dkgreen},
%   stringstyle=\color{mauve},
%   breaklines=true,
%   breakatwhitespace=true,
%   tabsize=3
% }

\begin{document}
\maketitle

\textbf{Note : In this course, notes for some parts of syllabus before midsem are missing}\\

\section{Introduction}

\begin{itemize}
	\item The five M's of the manufacturing define the profitability of the manufacturing process
	\begin{itemize}
		\item Men (women)
		\item Machine
		\item Materials
		\item Money 
		\item Management

	\end{itemize}

	\item Manufacturing processes can be classified into:
	\begin{itemize}
		\item Shaping processes \item alter the shape
		\begin{itemize}
			\item Solidification process
			\item Metal Removal process
			\item Deformation process
			\item Assembly process
		\end{itemize}
		\item Property enhancing processes \item change the properties of the material\\
		Ex: annealing, hardening
		\item Surface enhancing processes \item clean the surfaces or coat them\\
		Ex: Surface oxidation
	\end{itemize}


	\item Selection of manufacturing processes depends on various factors
	\begin{itemize}
		\item Cost 
		\item Available infrastructure
		\item Operating temperature
		\item Required properties
		\item Dimensions of required product
	\end{itemize}

\end{itemize}

\section{Casting Process}

	Pros:
	\begin{itemize}
		\item Complex forms, low cost
		\item Certain shapes cannot be machined
		\item One piece parts vs. multiple piece parts
		\item Design changes are easily incorporated
		\item High volume, low skilled labor
		\item Large, heavy parts can be made easily
	\end{itemize}

	\noindent Cons:
	\begin{itemize}
		\item Problems with internal porosity
		\item Dimensional variations due to shrinkage
		\item Trapped impurities, solids and gasses
		\item High-tolerance, smooth surfaces not possible
		\item More costly than stamping or extruding in some cases
	\end{itemize}

	\subsection{Pattern} 
	\begin{itemize}
		\item It is the replica (usually a bit bigger) of the req product 
		\item It should be cost efficient, easily machinable into common shapes.\\
		Possible choices are: 
		\begin{itemize}
			\item Wood : more than 90\% of the production happens with wood, absorbs moisture, relatively lower life
			\item Metal: longer life, used in large quantity production, Al, cast iron and bronze are most used
			\item Plastic : good corrosion resistance, smooth surfaces, do not absorb moisture, dimensionally stable, low weight
		\end{itemize}
		\item Pattern types
		\begin{itemize}
			\item Single piece pattern: Name says it all
			\item Split pattern
			\item Cope \item split pattern: Same as split pattern except that parts are moulded seperately
			\item Gated pattern: Tree like structure with each branch holding the required shape
		\end{itemize}
	\end{itemize}

	\subsection{Allowances}
	\begin{itemize}
		\item Shrinkage Allowance\\
			When the metal changes from liquid to solid state and cools downs in solid state, its volume reduces. So we make the pattern size more than the size required. This is called shrinkage allowance. 

		\item Draft/ Taper Allowance\\
			To avoid damage to the sand mould and casted metal on the internal walls while removing the metal after it cools, we need to provide extra material on the walls of the pattern. We typically use a tapering angle(angle at which pattern is bent outwards) of $1\deg$ to $3\deg$.

		\item Machining/Finishing allowance\\
			It is the extra material allowed to cut to smooth finish the product. The amount of machining depends on method of moulding method of molding and casting used, size and shape of casting, metal used in casting, required accuracy and finish

		\item Distortion / Camber allowance\\
		 	Due to typical shape of some patterns (non symmetric) (T, U, V, W), after cooling down the shape will change due to uneven shrinkage, so we need to allow distortion allowance i.e, we need to bend the pattern in opp dir to the expected distortion to get correct shape. This varies between 2 to 20 mm.
	\end{itemize}

	\subsection{Molds}
	\begin{itemize}
		\item Expendable Molds: A new mold has to be produced for every cycle. Types: green sand, dry sand, shell, investment, plaster
		\item Permanent molds: Multiple usages, used in: die casting, centrifugal, pressure die, injection molding

		\item Required properties of moulding sand:
		\begin{itemize}
			\item Refractoriness: Ability to withstand higher temperatures
			\item Permeability: For the air to escape
			\item Cohesiveness: Adhesive forces in sand
			\item Flowability: It should flow uniformly to all the portions
		\end{itemize}
	\end{itemize}


	\subsection{Gates}
		Gate is the hole through which the molten metal flows into the mould cavity. Mainly 3 types of gates are used:
	\begin{itemize}
		\item Top gating: The gate is present at the topmost portion of the cavity. In this case, the metal poured through the gate hit the mould cavity floor with some force. This can cause the sand to erode and mix up with the molten metal. But this can be avoided by making the passage from runner to gate a bit above the runner floor. This will slow the metal.

		\item Bottom gating: In this case, gate is present at the lowest part of the mould cavity. After filling some part of the mould cavity, the process slows down as the pressure difference driving the flow of metal is getting reduced.

		\item Parting line gating: Gate is present at parting line or joining line of cope and drag.

		\item A top gating design is prefered over bottom gating design and parting gating design because in the case of top gating design, the gravity help the molten metal to flow to intricate corners of the pattern.
	\end{itemize}

% todo : Core

\subsection{Types of Casting}

\begin{itemize}
	\item Sand casting:\\
		Advantages - This is cheap, this can be used ferrous and non-ferrous materials. (Non ferrorus metals require high temp)\\
		Disadvantages - Rough finish

	\item Investment casting
		\begin{itemize}
			\item We first create a pattern out of wax(optimisied for this process)  and then dip it in slurry (silica). The slurry solidify on the outer surface to req thickness. Then we heat the thing to melt wax and remove the wax completely to create a outer wall for the req shape. The liq metal is poured into the solid slurry to create the req shape. After cooling down, outer shell can be removed
			\item This has a smooth wall finish
			\item It can be used to create complex shapes because wax can be easily bended 
			\item Also known as lost wax casting precision casting
			\item This is relatively expensive compared to sand casting
		\end{itemize}


\item Centrifugal casting
	\begin{itemize}
	\item Molten metal is poured into a long cylinder and rotated, and it is cooled on the outer surface
	\item The outer part cools faster and have smaller grain size
	\item The lighter impurities present in the metal get accumulated at the inner surface. This can be removed by machining process in the inner surface
	\end{itemize}

	
\item Shell moulding
	\begin{itemize}
	\item The req shape in metal in heated and some silica or sand made to stick to the metal and a outer shell is made.
	\item Then liquid metal is poured into the mould.
	\item Axial symmetry things can be made easily by this.
	\end{itemize}


\item Lost Foam/ Evaporation Pattern
	\begin{itemize}
	\item A pattern is created is using polystryene and sand is used to support it.
	\item The metal is directly poured into the polystryene and the polymer evaporates and creating req shape
	\end{itemize}


\item Continuous casting
	\begin{itemize}
	\item Partially solidified (outer surface) metal is sent through the roller and cooling liquid is sprayed
	\item This creates uninterupted long strands of metal
	\end{itemize}


 \item (Pressure) Die casting
 	\begin{itemize}
 	\item High cost
 	\item The dies are two shapes which lock into each creating a cavity in between. 
 	\item Before locking they are sprayed with some liq to prevent sticking of molten metal to die
 	\item Metal is poured is cavity and pressure is applied to molten metal. Pressure ensures that metals reaches all the cavities and ensures that metal reaches everywhere before it start solifidies. After removing ejector pins, dice get seperated.
 	\item Cooling channels are present in the die which cool the metal. The whole can be completed as fast as 1min.
 	\item In cold chamber die casting - higher melting point metals (prefered because it can cost higher to maintain metal in liq state) 
 	\item In hot cold chamber die casting - lower melting metal (can also be used for higher melting metal)
	\end{itemize}


\end{itemize}


\subsection{Casting Defects}
\begin{itemize}
 	\item Misrun : metal solidified before reaching all places
 	\item cold shut : meeting of two metal streams is not complete, i.e, metal flow was not able to reach parts of the mould cavity where two metal flows meets
 	\item cold shot : Due to turbulence during the pouring of the metal, some solid globules form which become entraped in the metal. 
	\item Sand blow : The gas bubble in the cavity which entered the during pouring (or gases released by metal as it cools) stays in the cavity and create bubble shaped cavity.
	\item Pinholes : The small pours in the metal which are due to the gases which are released during solidification
	\item Blowholes : Blowholes are similar to pinholes but with larger holes.
	\item Shrinkage cavity: Depression or internal void created due to shrinkage of metal volume while cooling is called shrinkage cavity
	\item Hot tearing : when the casting is restrained from contraction by an unyielding mold during the final stages of solidification or early
	\item Sand wash : Irregularity in the surface of the casting that results from erosion of the sand mold during pouring
\end{itemize}


\subsection{Trends}
% (todo : more on this)
\begin{itemize}
	\item Change in casting metal
	\begin{itemize}
	 	\item Steel to ductile iron - lower material and conversion cost
	 	\item Al to Al-Mg-Ti alloys - increase in stength to weight ratio
	 	\item In alloy casting, we need to ensure that the composition remains the same throughout.
	 \end{itemize}
	\item Change in geometry
	\begin{itemize}
		\item Wall thickness is reducing. This is because in casting of thick pieces, distortion happens and we want straight parts. In case, we have multiple thickness, then we can use chill to start solidification earlier at thicker points
		\item Weight reduction and increasing shape complexity
	\end{itemize}
\end{itemize}



\section{Metal flow in typical casting}
\begin{itemize}
	\item Metal flow is turbulent. Turbulence causes mixing with air, sand erosion	
	\item When metal flows against the mold wall erosion, vortex formation (air pocket/slag), splashing against the mold walls, air aspiration(due to reaction with mold material) and metal solidification.
	\item We can determine the reynold's numbers of molten metal.
	\item Velocity of the metal gets reduced due to loss by sudden change in cross section, etc.
		\[v_{actual}=c_d \sqrt{2gh}\]
	\item Typical value of $c_d$ is 0.6-0.8
	\item Friction can be ignored for considering loss in velocity of metal, losses due to change in cross section dominate the losses
	\item Sprue shape has to be shaped as a fructum pointing downwards because as the velocity of the metal increases (due to gravity), the area of cross section gets reduced ($A_1 v_1 = A_2 v_2$). If the shape is cylindrical or fructum pointing upwards, the flow seperates from the mould walls and form a air cavity which can lead to defects. 
	\item For multiple top gates, L gate (gate at the end of the sprue) has higher $c_d$ compared to T gate. L gate has the highest flow rate (even if it is the farthest) due to inertia.
	\item To reduce defects in casting process due to erosion of sand mould, we need to ensure that \\
	Stress due to impact $>$ Sand mould strength
	\item Velocity of the metal should not be very high, because high vel makes erosion and causes defects in metal product
	\item Desired $Re <8000$
	\[Re = \frac{\rho v D}{\mu}\]

	\item Surface turbulence (Jetting or fountaining effect) - Weber number \\
	($\gamma$ is the surface tension of the metal)
	\[W_e = \frac{\rho v^2 d}{\gamma}\]
	$W_e <1$: No surface turbulence\\
	$W_e <10$: Mild \\
	$W_e <100$: High turbulence

	\item Bond number (Related to buoyancy)
	\[B = \frac{\rho d^2 g}{\gamma}\]
	$B <1$: Surface tension dominate\\
	$B>1$: buoyancy force overcomes surface tension and leads to surface instability
\end{itemize}

	\subsection{Fluidity}
	\begin{itemize}
		\item \textbf{Fluidity} in case of metal casting refers to the distance the metal can travel before solidifying. This is different from the physics defination(inverse of viscosity)
		\item Fluidity can be increased by increasing pressure and/or pouring temperature.
		\item No of gates has to be increased if the fluidity is low
		\item Pure metals have higher fluidity than alloys

		\item Spiral test / honeycomb test can be carried out to find the fluidity. This can also be simulated using finite element analysis software like Ansys etc.
		\item This also can be roughly estimated by doing some calculations:\\
		Consider a metal at its melting point $T_m$ poured into a channel of radius a with average flow velocity v. Let the metal solidify at a distance $L_f$ in time t. The heat released by metal (latent heat) per unit time is equal to the heat conducted by the mold-metal interface. Let h be the heat transfer coefficient of mold-metal interface. Then
		\[\rho_m \pi a^2 V \Delta H = (2 \pi a L_f) h (T_m - T_0)\]
		\[{L_f = \frac{\rho_m v \Delta H a}{2 h (T_m - T_0)}}\]
		\item The above expression is valid only for pure metals since the melting point of alloys is not a single point but a range of temperatures

		\item Fluidity of alloys when solid grains are present can be expressed as
		\[L_f = f_s^{cr} V t_s\]
		where $f_s^{cr}$ is the critical fraction of solid at which fluid stops
	\end{itemize}

	\subsection{Aspiration effects in mold filling}
	\begin{itemize}

		\item For a mold made up of permeable material (eg sand), the pressure of the metal stream should not be less than atmospheric pressure. Because if this happens gases from the baking of organic compounds in mold will enter the molten metal stream, producing porous casting. This is known as \textbf{aspiration effect}.
		\item Consider a sprue (part between pouring basin and gate) of uniform diameter of height h which lead to a mould cavity open to atmosphere (completely or partially through runner). Let the pressure at the start of the sprue be P and velocity $v_1$. The pressure at the end of the sprue is $P_{atm}$ (since it is open to atmosphere) and let the velocity be $v_2$.\\
		Since the sprue diameter is uniform from continuity equation, we have $v_1 = v_2$.\\
		By applying Bernouli principle,
		\[P + \rho g h = P_{atm}\]
		\[P = P_{atm} - \rho g h\]
		Here, the pressure is less than the atmospheric pressure which causes aspiration. 

		% \item To avoid this we need make the cross section area of the lower part of the sprue $A_2$ less than the area of upper part $A_1$ such that it compensates for the change in height.\\
		% Let the $A_2/A_1 = R$, (R<1)

	\end{itemize}


\section{Joining, Assembly and Welding}
	% - Rivet joint:
	% 	- It is permanent joint.
	% 	- In this, a pin is inserted and pulled so as to break it inside t

	\subsection{Joining}
		\begin{itemize}
			\item Welding, brazing and soldering and adhesive are permanent joints
			\item Welding is with metal of low temp and brazing is with metal of high temp
		\end{itemize}

	\subsection{Welding}
		\begin{itemize}
			\item Two or more parts are coalesced at their contacting surfaces using heat or pressure. 
			\item We can (depends on metals we use) use another metal to facilate joining of two parts. The external metal should have high solubility with the parts being joined.
			\item \textbf{Pros}:
				\begin{itemize}
					\item Welding is usually the most economical way to join parts
					\item Can be done outside factory environment
				\end{itemize}
			\item \textbf{Cons}:
				\begin{itemize}
					\item This is usually done manually, so is costly
					\item Cannot be easily disassembled
				\end{itemize}

			\item Welding can be performed in various environment - vaccuum (electron beam welding), normal atmosphere, underwater.
			\item Fusion welding: We melt the base metals and joined to create a homogenous joint. We can use a filler metal here to bond the two parts. In this, we can provide inert gas environment known as shieding gas. 
			\item Solid state welding: No molten states. One of them is friction stir welding. No molten state prevents the formation of blowholes and pinholes because they do not absorb large amount of gases in molten which is the case in fusion welding. Increasing becoming popular

			\item {Arc welding (Fusion welding)}: We create an electric arc between electrode and work piece (we bring them close but do not let them touch and apply high voltage between them). This heats up the electrode and the metal melts which acts filler. We need to use shielding gases to prevent absorbtion of gases by metal.
			\item Friction stir welding: Currently used for aluminium (this can easily get oxidised in fusion welding). In this process, we insert a pin between the parts to be joined and stir it at high speed and  to generate heat moving throught the joint, which will join the parts
			\item Laser welding : This is type of fusion process in which we use laser to melt the metal.

		\end{itemize}
		

	\subsection{Fusion welding}
		\begin{itemize}
			\item Shielding: When metal is in molten state, its gas adsorbtion capacity is very high which leads to pinholes and blow holes. To prevent this, a inert gas environment is provided. Usually, when the metal in the electrode is consumed, it releases some inert gases which creates shielding.
			\item Popular welding in this category
				\begin{itemize}
					\item GTAW - Gas Tungsten arc welding
					\item GMAW - Gas 
					\item SMAW
					\item OAW - oxy acetylene welding
				\end{itemize}

		\item \textbf{Arc welding}:
			\begin{itemize}
				\item We need to maintain a constant optimal gap to maintain the electrical arc.
				\item A human cannot perform this very well, so we use machines to perform this. Machines cna do this very efficiently.
			\end{itemize}

		\item \textbf{Resistancee welding}:
			\item We put the two metal parts in a circuit with high current. Utilisng joule's heating, we melt the metal. The metals are also put under proessure.

		\item \textbf{Oxy aceltylene welding}: We use the oxyaceltylene flame to melt the parts and join them. IN this welding, a lot of mechnaical pressure is not applied
		\end{itemize}

	\subsection{Solid state welding}
		\begin{itemize}
			\item Diffusion welding: We make the two part surface flat and put them together under pressure. We heat the combination under pressure. At high pressure, the area of contact increases and increasing temperature increases diffusion rate. So after some time, the metals have formed a homogenous bond. Usually this is for two parts of same metal
		\end{itemize}


% \item Fillet weld:
% 	\item Used to fill the edges of plates created by corner, lap and tee joints.
% 	\item Using oxyfuel welding and arc welding.
% 	\item Filler metal is approximately filled in a triangualer crosssection.

	\subsection{Parameters to evaluate welding}
		\begin{itemize}
			\item \textbf{Power density}: Power supplied per unit surface area.
				\begin{itemize}
					\item If it is too low, heat will be conducted and required temperature will not be achieved.
					\item If it is too high, it will vapourise the metal
					\item There are some practical ranges for this. 
					\item For example, in arc welding, it is focused and weld can occur upto alot of depth but in the gas welding, it is no the case
					\item Oxyfuel welding(10 $Wmm^{-1}$) can produce high temperature  but as it is spread over alot of area. upto $3500^\circ$,
					\item Laser beam welding(9000$Wmm^{-1}$) / electron beam welding (10000 Wmm-1)have a very high pwoer density
					\item Arc weldding - 50$Wmm^{-1}$
					\item It has to be high for thick weld
				\end{itemize}
			\item \textbf{Welding speed}: It is high for electron beam but for gas welding it is low. We need to find a optimal speed.

			\item \textbf{Aspect ratio}\\
				$\text{Aspect ratio} = z/(x,y)$\\
				For high aspect ratio, power density has to high

			\item Unit energy for melting:
				\item $f_1$ 	Heat received by metal/ HEat given by source (Heat transfer efficiency)
				\item $f_2$ Heat received by melt /Heat given to metal (Melting efficiecy)
				\item Heat used $ H_w = f_1 f_2 H $
				\item Al and Cu present a problem because it has high thermal conductivity. It dissipates a lot of heat that it receives.
				\item Energy balance equation : Heat delivered $H_w = U_w V$, where $U_w$ is the heat required per unit volume , V is volume

			\item \textbf{Joining efficiency}:\\
				$\text{Joining Effiecieny} = vt/P$\\
				v = trasverse speed mm/s\\
				t = thickness of weld mm\\
				P = incident power, kW\\

			% 	Type 		Size of welding
			% TIG welding 	Large to medium
			% MIG weldding 	Large to medium
		\end{itemize}
	
\pagebreak


\section{Metal Forming}
	\begin{itemize}
		\item In many manufacturing processes, plastic deformation is used to shape the work piece.
		\item A tool (usually called die) applies stresses more than the yield stress of the metal and less than the fractures strength of the material.
		\item Stresses used to deform the work piece are usually compressive. Tensile stresses may also applied be on the work piece to stretch the metal. 
		\item \textbf{Sheet Metal working} : In sheet metal working, the ratio $\frac{V}{S}$ is lower (thickness is low), i.e, it is typically done on metal sheets, strips or coils. 
		\item Desirable material properties (for metal forming):
			\begin{itemize}
				\item Low yield strength (easy to deform)
				\item High ductile
			\end{itemize}
	\end{itemize}

	\subsection{Bulk Deformation Processes}
		\begin{itemize}
			\item Bulk refers to point that the ratio $\frac{V}{S}$ is higher (thickness is higher). 
			\item Rolling : Plate is squeezed between the rolls
			\item Forging : Work is squeezed between opposing dies
			\item Extrusion : sqeezed through a die opening. Workpiece takes the shape of the opening.
			\item Wire and bar drawing : Diameter is reduced by pulling through the die opening.
			\item If the working temperature is below recrystallation temperature, it is called cold forming and if the working temperature is more than the recrystalisation temperature then it is called hot forming. 
			\item If the direction of the flow of material and the piston is same, then the extrusion is called direct extrusion, if it is opposite, it is called indirect extrusion.
		\end{itemize}

	\begin{itemize}
		\item \textbf{Strain hardening}: Strain Hardening is when a metal is strained beyond the yield point. An increasing stress is required to produce additional plastic deformation and the metal apparently becomes stronger and more difficult to deform. This is mainly seen in cold forming process. 

		\item Behaviour in metal forming:
			\begin{itemize}
				\item In plastic region, metal's behavior is expressed by the flow curve, 
					\[\boxed{\sigma_f = K \epsilon^n}\]
				where $\sigma_f$ = flow strength, K = strength coefficient and n = strain hardening exponent. 
				\item Average flow stress (${\bar \sigma_f}$)
					\[{\bar{\sigma_f}= \frac{\int_0^{\epsilon_f} K \epsilon^n d\epsilon}{\epsilon_f} = \frac{K \epsilon_f^n}{n+1}}\]
				where $\epsilon_f$ = maximum strain during deformation
			\end{itemize}

		\item Strain Rate:
			\begin{itemize}
				\item As the strain rate increases, yield strength increases. 
				\item At room temperature, the effect of strain rate is negligible. 
			\end{itemize}

		\item \textbf{Recrystallization temperature}: Temperature at which recrystallization just reaches completion in 1 h.
		\[0.3T_m < T_R < 0.6T_m\]
		\item \textbf{Annealing} is a heat treatment that alters the physical and sometimes chemical properties of a material to increase its ductility and reduce its hardness, making it more workable.

		\item Effect of temperature:
			\begin{itemize}
				\item Temperature ranges for deformation:
					\begin{enumerate}
						\item Cold working : Below recrystallisation temperature ($<0.3 T_m$)
						\item Warm working $(0.3 T_m - 0.5 T_m$)
						\item Hot working : Above recrystallisation temperature ($0.5 T_m - 0.7 T_m$)
					\end{enumerate}
				\item As temperature increases, ductility increases and yield strength decreases. 
				\item At higher temperature, the effect of roughness of roll or die increases, ie., since the metal is soft at higher temperature the roughness on the die or roll gets printed on the work piece. 
			\end{itemize}

		\item \textbf{Cold working} : 
			\begin{itemize}
				\item Performed at room temperature or slightly above. 
				\item Important for mass production
				\item No machining is required
				\item Strain hardening is dominant and strength increases.
				\item Pros:
					\begin{itemize}
						\item Better accuracy
						\item No heating required
						\item Strain hardening increases strength
					\end{itemize}
				\item Cons:
					\begin{itemize}
						\item High forces required
						\item Ductility and strain hardening limit the deformation that can be performed
					\end{itemize}
				\item \textbf{Percentage of cold working}
				\[\text{\% Cold working }= \frac{Initial Area - Final Area}{Initial Area} \times 100\%\]
				\item As cold working increases, yield strength increases and ductility decreases. 
			\end{itemize}

		\item \textbf{Hot working}:
			\begin{itemize}
				\item Performed above recrystallisation temperature
				\item Capability of hot working to perform deformation is lot more than cold working
				\item Pros:
					\begin{itemize}
						\item Low forces and power required
						\item Large amount of deformation can be carried
						\item Strength properties are generally isotropic
					\end{itemize}
				\item Cons:
					\begin{itemize}
						\item Work piece suseptible to oxidation due to higher temperature
						\item Shorter tool life
					\end{itemize}
			\end{itemize}

		\item If perform annealing on cold worked work piece (heat the material slowly), the ductility increases and tensile strength of the material decreases. As the temperature increases, the elongated grains dissolve to form new grains which combine to grow bigger grains. 
	\end{itemize}

		\subsection{Rolling}
		\begin{itemize}
			\item Types of rolling:
			\begin{itemize}
				\item Flat rolling: used to reduce thickness
				\item Shaping rolling : used to shape the workpiece, for instance, into I-shaped beam
			\end{itemize}

			\item Types of rolling
			\begin{itemize}
				\item Hot rolling: large of deformation is required
				\item Cold rolling : Dimensional accuracy and finishing is important
			\end{itemize}

			\item Rolling mill configurations:
			\begin{itemize}
				\item Two-high : Two large opposing rolls
				\item Three-high : work passes through rolls in both directions
				\item Four-high : Two large rolls supporting the smaller rolls
				\item Cluster mill – multiple backing rolls on smaller rolls
				\item Tandem rolling mill – sequence of two-high mills
			\end{itemize}

			\item \textbf{Flat rolling}:
			\begin{itemize}
				\item Contact length: The length of the work piece that is in contact with the material
				\item Bite angle: The angle subtended by contact length at center of the roll.
				\item Using geometry, we can derive that
				\[R(1-\cos\theta) = \frac{t_i - t_f}{2}\]
				\item As the work piece rolls through the roller, the velocity of the work piece increases since its volume remains the same but the length increases. 
				\[v_0<v_r<v_f\]
				\item There is a point on the roller whose velocity is equal to the roller velocity. This point is called neutral point.
				\item Smaller-diameter rolls produce less contact length for a given reduction and therefore require lower force and less energy to produce a given change in shape but they are also more suseptible to deformation themselves. Hence, they supported by larger rolls.
			\end{itemize}
			
			\item \textbf{Thread rolling}: 
			\begin{itemize}
				\item Die on which thread like intrusions are made is used to thread like structures on the bolts and and screws. 
				\item They are produced faster than the machining process using rolling.
			\end{itemize}

			\item \textbf{Ring rolling}: 
			\begin{itemize}
				\item In this a ring of smaller diameter is rolled into a thin-walled ring of larger diameter.
				\item The ring is passed between a idler roller and a main roller which apply stresses to elongate it, increasing the diameter. 
				\item This is used to produce steel tires for railroad wheels, rings for pipes, pressure vessels, and rotating machinery.
			\end{itemize}
		\end{itemize}

		\subsection{Failure theories}:
			Let $\sigma_1\geq \sigma_2 \geq \sigma_3$ be the principal stresses and uniaxial yield stress be $\sigma_Y$.\\
			Maximum shear stress = $\frac{\sigma_1 - \sigma_3}{2}$.\\
			Maximum shear in uniaxial tensile test = $\frac{\sigma_Y}{2}$

			\begin{enumerate}
				\item \textbf{Maximum principal stress theory} (Rankine' theory - for brittle materials): \\
					Material will fail if maximum principal stress exceeds maximum stress in unaxial tensile test.
					\[\boxed{\sigma_1 \geq \sigma_Y}\]
				\item \textbf{Maximum shear stress theory} (Teresca theory - for ductile materials): \\
					Material will fail if maximum shear stress exceeds maximum shear stress in unaxial tensile test.
					\[\boxed{\frac{\sigma_1 - \sigma_3}{2} \geq \frac{\sigma_Y}{2}}\]

				\item Maximum principal strain theory (St Venants) :\\
				If the maximum principal strain exceed strain in uniaxial tensile test, then the material will fail i.e, if
				\[\epsilon_Y = \frac{1}{E}(\sigma_1 - \nu (\sigma_2 +\sigma_3)) \geq \frac{\sigma_Y}{E}\]

				\item Total strain energy theory:\\
					The material fails if the strain energy with the principal stresses exceed the strain energy from the uniaxial tensile test, i.e, 
				\[\frac{1}{2E}[\sigma_1^2 + \sigma_2^2 + \sigma_3^2 - 2\nu(\sigma_1 \sigma_2 + \sigma_2 \sigma_3 + \sigma_3 \sigma_1)] > \frac{1}{2E} \sigma_Y^2\]

				\item \textbf{Distortional theory }(Von-Mises theory - for ductile materials):\\
					 \[\boxed{(\sigma_1-\sigma_2)^2 + (\sigma_2 -\sigma_3)^2 + (\sigma_3-\sigma_1)^2 \geq 2 \sigma_Y^2}\]
			\end{enumerate}



\end{document}