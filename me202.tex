\documentclass{article}
\title{ME 202 - Strength of Materials}
\author{Vishal Neeli}
\date{}

\usepackage[a4paper, total={6in, 11in}]{geometry}
\usepackage{textcomp}
\usepackage{hyperref}
\usepackage{amsmath}
\usepackage{xcolor}
\usepackage{grffile}
\usepackage{physics}
\hypersetup{
	colorlinks=true,
	urlcolor=blue,
	linkcolor=cyan,
	filecolor=red
}
\usepackage{amsfonts}

% FOR CODE
% \usepackage{listings}
% \usepackage{color}

% \definecolor{dkgreen}{rgb}{0,0.6,0}
% \definecolor{gray}{rgb}{0.5,0.5,0.5}
% \definecolor{mauve}{rgb}{0.58,0,0.82}

% \lstset{frame=tb,
%   language=Java,
%   aboveskip=3mm,
%   belowskip=3mm,
%   showstringspaces=false,
%   columns=flexible,
%   basicstyle={\small\ttfamily},
%   numbers=none,
%   numberstyle=\tiny\color{gray},
%   keywordstyle=\color{blue},
%   commentstyle=\color{dkgreen},
%   stringstyle=\color{mauve},
%   breaklines=true,
%   breakatwhitespace=true,
%   tabsize=3
% }

\begin{document}
\maketitle

\section{Review}
	\subsection{Strain Energy}
	\begin{align*}
		u = \frac{1}{2} E \epsilon^2 = \frac{1}{2} \sigma \epsilon = \frac{1}{2} \frac{\sigma^2}{E}
	\end{align*}

	\subsection{Poison's Ratio}

	$\epsilon_{axial}= \frac{\delta}{L}$, $\delta$ is the change in longitudinal length\\
	$\epsilon_{radial}= \frac{\delta'}{r}$, $\delta'$ is the change in radius\\

	Poisson's ratio -
	\[\nu = -\frac{\epsilon_{radial}}{\epsilon_{axial}}\]
	\[0\leq \nu \leq 0.5\]

	Shear stress and strain
	\[\tau = G \gamma\]
	Relation between G, E and $\nu$ - 
	\[G= \frac{E}{2(1+\nu)}\]

	For a axial load, 
	\[d\delta = \frac{N(x) dx}{A(x) E(x)}\]


	\subsection{Stresses in a Plane and Mohr's Circle}
	Using physical equations or using cauchy's stress tensor with rotation matrix, we get
	\begin{gather}
		\sigma_{x'} = \frac{\sigma_x + \sigma_y}{2} + \frac{\sigma_x - \sigma_y}{2} cos2\theta + \tau_{xy} sin2\theta \\
		\tau_{xy'} = -\frac{\sigma_x - \sigma_y}{2} sin2\theta + \tau_{xy} cos2\theta \\
		\sigma_{y'} = \frac{\sigma_x + \sigma_y}{2} - \frac{\sigma_x - \sigma_y}{2} cos2\theta - \tau_{xy} sin2\theta
	\end{gather}

	The above equations can be viewed as parametric equations to a circle.

	On squaring and adding, we get Mohr's circle

	\[\left(\sigma_{x'} -\frac{\sigma_x+\sigma_y}{2} \right)^2 + \tau_{xy'}^2 = \left(\frac{\sigma_x - \sigma_y}{2}\right)^2 + \tau_{xy}^2 \] 

	Center = $(\frac{\sigma_x+\sigma_y}{2},0)$ and 
	Radius = $\sqrt{\left(\frac{\sigma_x - \sigma_y}{2}\right)^2 + \tau_{xy}^2}$\\

	\begin{itemize}
		\item This is plotted by taking $\sigma_{x'}$ on x axis and $\tau{xy'}$ on \textbf{negative} y-axis and angle $2\theta$ anticlockwise as positive.\\
		\item $\tau{xy}$ is taken as positive if it tends to rotate in the anticlockwise direction, negative otherwise.
		\item $\sigma_x$ is taken as positive if it is tensile and negative for compressive.
		\item A rotation of angle $\theta$ in the plane corresponds to a rotation of $2\theta$ on the Mohr's circle.
	\end{itemize}


	\subsection{Hooke's law for plane stresses }
	For a point (isotropic material) under planar stresses ($\sigma_{xz}=\sigma_{yz}=\sigma_{zz} =0$), we have
	\begin{gather*}
		\epsilon_x = \frac{1}{E} (\sigma_x - \nu \sigma_y)\\
		\epsilon_y = \frac{1}{E} (\sigma_y - \nu \sigma_x)\\
		\gamma_{xy} = \frac{\tau_{xy}}{G}\\
		% \left{\text{Using these,}}
		\sigma_x = \frac{E}{1-\nu^2} (1+\nu \epsilon_x)\\
		\sigma_y = \frac{E}{1-\nu^2} (1+\nu \epsilon_y)\\
		\tau_{xy} = G \gamma_{xy}
	\end{gather*}



	\subsection{Volume change and Strain-energy density}
		Let a cuboid of sides a,b and c be under stresses. 
		\[V_0 = abc\]
		\begin{align*}
		V &= (a+a\epsilon_x) (b+b\epsilon_y)  (c+c\epsilon_z)\\
		  &= abc(1+\epsilon_x) (1+\epsilon_y)  (1+\epsilon_z)\\ 
		  &= V_0 (1+\epsilon_x +\epsilon_y + \epsilon_z ) &\text{ignoring the $\epsilon_x\epsilon_y$ terms}
		\end{align*}

		Unit volume change e, also known as \textbf{dialatation}\\
		\[e=\frac{\Delta V}{V_0} = (1+\epsilon_x) (1+\epsilon_y)  (1+\epsilon_z)\]

		Strain-energy density in plane stress,
		\begin{align*}
			u &= \frac{1}{2}(\sigma_x\epsilon_x + \sigma_y \sigma_y + \tau_{xy} \gamma_{xy})\\
			u &= \frac{1}{2E} (\sigma_x^2 +\sigma_y^2 - 2 \nu \sigma_x \sigma_y +\frac{\tau_{xy}^2}{2G})\\
			u &= \frac{E}{2(1-\nu^2)}(\epsilon_x^2 +\epsilon_y^2 +2\nu\epsilon_x \epsilon_y)+\frac{G\gamma_{xy}^2}{2}
		\end{align*}
	\subsection{Hooke's law for triaxial stresses}
	($\sigma_z=\tau_{xz}=\tau_{yz}=0$)
	\begin{gather*}
		\epsilon_x = \frac{1}{E}(\sigma_x -\nu \sigma_y -\nu \sigma_z)\\
		\sigma_x = \frac{E}{(1+\nu)(1-2\nu)}[(1-\nu)\epsilon_x + \nu (\epsilon_y+\epsilon_z)]
	\end{gather*}


	\subsection{Strain Energy in Torsion}
	\begin{gather*}
		d\phi = \frac{T dx}{GJ}
	\end{gather*}
	\begin{align*}
		\text{Energy stored due to torsion, } U &= \int_0^L \frac{1}{2} T d\phi\\
											    &= \int_0^L \frac{T^2 dx}{2GJ} \\
	\end{align*}


\section{Torsion}
\label{torsion}
	\begin{itemize}
		\item \textbf{Angle of twist} : It is the angle by which one end of the rod is displaced wrt the other end of the rod under the effect of some torsion (twisting).
		\item Consider a small cylidrical section which has length $dx$, one end has been displaced wrt another by an angle of $d\phi$. Consider a line $ab$ on the circumference along the length of the cylinder which has be changed to $ab'$. Then shear strain 
		\begin{equation*}
			\gamma_{max} = \frac{bb'}{ab} \qquad (\text{bb' can be assumed to be a straight line})\\
		\end{equation*}
			\[\gamma_{max} = \frac{r d\phi}{dx}\]


		\item Rate of twist or angle of twist per unit length
		\[\theta = \dv{\phi}{x}\]
		\[\gamma_{max} = r \theta\]
		If $\theta$ is constant, then
		\[\gamma_{max}= \frac{r \phi}{L}\] 
		This is called as $\gamma_{max}$ because we are measuring the shear strain at the outer end i.e, with maximum radius and hence, maximum shear strain.
		\[\boxed{\gamma = \rho \theta = \rho \dv{\phi}{x}}\]
		where $\rho$ is the perpendicular distance of the point from the axis (radius) we are considering.
		\[\boxed{\gamma = \frac{\rho}{r} \gamma_{max} }\]

	\end{itemize}
	\subsection{Hooke's Law}
		Hooke' law for shear stress and shear strain
		\begin{align*}
			\tau = G \gamma
				 = G \frac{\rho}{r}\gamma_{max}
				 = \frac{\rho}{r}\tau_{max}
		\end{align*}
		\[\tau_{max} = G \gamma_{max}\]


	\subsection{Torsion Formula}
	\begin{itemize}
		\item Polar moment of inertia (this is integral over area - double integral)-
		\[\boxed{I_P = \int_A \rho^2 dA}\]
		where $\rho$ is the distance at which area element $dA$ is located.
		\item For a circle of radius r, 
		\[I_P = \int_A \rho^2 dA = \int_{\theta = 0}^{2\pi}\int_{\rho = 0}^r \rho^2 (\rho d\theta d\rho) = \frac{\pi r^4}{2} \]
		\[\boxed{I_{Pcircle} = \frac{\pi r^4}{2}}\]
		\item Consider a cross-section of any shape, we are trying to sum all the small torques and equate it to the torque applied on this
		\begin{align*}
			T = \int_A dM &= \int_A \rho \tau dA\\
						  &= \int_A \frac{\rho^2}{r}\tau_{max} dA\\
						  &= \frac{\tau_{max}}{r} \int_A \rho^2 dA\\
						T &= \frac{\tau_{max}}{r} I_P
		\end{align*}
		Or,
		\[\tau_{max} = \frac{Tr}{I_P}\]
		\item The shear stress at distance $\rho$ from the center of the bar with polar moment of inertia $I_P$ is
		\[\boxed{\tau = \frac{T\rho}{I_P}}\]
		So, $I_P$ represents the resistance to change in twist angle (or shear stress) by virtue of its cross section.\\
		\emph{Note}: Here, the $I_P$ is constant for any distance $\rho$.\\
		For a rod of \textbf{circular cross section} of radius r,
		\[\tau_{max} = \frac{2T}{\pi r^3} = \frac{16T}{\pi d^3 }\]

		\item Rate of twist
		\[\theta = \frac{T}{GI_P}\]
		Hence, $GI_P$ is also known as \textbf{Torsional Rigidity}
		\item For a bar in pure torsion ($\theta = const$), the total angle of twist
		\[\boxed{\phi = \frac{TL}{GI_P}}\]
		The quantity $\frac{G I_P}{L}$ is also known as \textbf{torsional stiffness} of the bar.

		\item For a thin circular tube, $I_P = 2 \pi r^3 t$
	\end{itemize}

	\subsection{Non-uniform Torsion}
		For a bar with non-uniform cross-section, tension  :
		\[\boxed{\phi = \int_0^L \frac{T(x) dx}{G I_P(x)}}\]


	\subsection{Torsion Formula for non-prismatic bars}
	\begin{itemize}
	\item For elliptical cross section, maximum shear stress
		\[\tau_{max} = \frac{2T}{\pi ab^2}\]
		\[\phi = \frac{TL}{GJ_e}\]
		\[J_e = \frac{\pi a^3 b^3}{a^2 + b^2}\]


	\item For triangular cross section
		\[\tau_{max} = \frac{T \frac{h}{2}}{J_t}\]
		\[\phi = \frac{TL}{GJ_t}\]
		\[J_e = \frac{h_t}{15\sqrt{3}}\]

	\item For rectangular cross section
		\[\tau_{max} = \frac{T}{k_1 b t^2}\]
		\[\phi = \frac{TL}{(k_2bt^3)G}= \frac{TL}{J_rG}\]
		\[J_r = k_2 b t^3\]
		$k_1$ and $k_2$ are emphirically determined and are dependent on $\frac{b}{t}$

	\item Thin walled open cross sections
	We treat flange and web as seperate rectangles.
		\[J = J_w + 2J_f\]
		\begin{gather*}
		J_f = k_2 b_f t_f^3 \\
		J_w = k_2 (b_w-2t_f) t_w^3\\
		\tau_{max} = \frac{2T (\frac{t}{2})}{J}\\
		\phi = \frac{TL}{GJ}
		\end{gather*}
	\end{itemize}

	\subsection{Thin walled tubes}
		Consider a small element of length of $dx$, then shear stresses are $\tau_a$,$\tau_b$, $\tau_c$ and $\tau_d$. Then the shear stresses on opposite wall should be equal (to satisfy Newton's second law). Hence,
		\begin{align*}
			F_a &= F_c\\
			\tau_a t_a dx &= \tau_c t_c dx\\
			\tau_a t_a &= \tau_c t_c
		\end{align*}

		Shear flow f,
		\[\boxed{f = \tau t = const}\]

	Deriving torsion formula for the thin walled tubes- \\
	($L_m$ is the total circumferencial length, $A_m$ is the area enclosed by the median line, r is the distance of the element to the center)

	\begin{align*}
		T &= \int \tau dA r\\
		  &= \int_0^{L_m} frds\\
		  &= 2 f A_m 
	\end{align*}

	\[\boxed{\tau= \frac{T}{2tA_m}}\]

	% Using shear stress, we get $J$ -\\
	Shear energy density of a solid under pure shear stress is $\frac{\tau^2}{2G}$. 
	\begin{align*}
		\text{Total shear energy }U &= \int\int \frac{\tau^2}{2G} t dx ds\\
								   &= \int \int \frac{f^2}{2Gt} dx ds\\
								   &= \frac{f^2}{2G} \int_0^L dx \int_0^{L_m} \frac{ds}{t}\\
								   &= \frac{T^2 L}{2G A_m^2} \oint \frac{ds}{t}
	\end{align*}
	Also, $U = \frac{T^2 L}{2G J}$.
	Hence, 
		\[\boxed{J = \frac{4A_m^2}{\oint\frac{dS}{t}}}\]
	For tube with \textbf{constant thickness},
		\[\boxed{J = \frac{4A_m^2t}{L_m}}\]

	For circular tube,
		\[J= 2\pi r^3 t\]

	For rectangular tube, 
		\[J = \frac{2b^2 h^2 t_1 t_2}{bt_1 + ht_2}\]\\

	Angle of twist,
	\begin{align*}
		\phi &= \frac{TL}{GJ}\\
			 &= \frac{TL}{4GA_m^2}\oint \frac{ds}{t}
	\end{align*}

	\[\boxed{\phi = \frac{TL}{4GA_m^2}\int_0^{L_m} \frac{ds}{t}}\]


	\subsection{Torsional stress concentration}
	\begin{itemize}
		\item If there is a sudden discontinuity in the cross section such as holes, , then local torsional stresses will develop and usual torsional formula cannot be applied.
		\item The maximum stress found near the discontinuity can be found using
		\[\tau_{max} = K \frac{Tc}{J}\]
		where K is the torsional stress concentration factor, r is the fillet radius and c and J are the radius and polar moment of inertia of the \textbf{smaller section} at the shaft.
		\item K values for various values of $\frac{r}{d}$ and $\frac{D}{d}$ are empirically (or using advaced theory) are determined and tabulated, where r is the fillet radius, d is the smaller diameter and D is the larger diameter
		\item Ref: \href{https://classes.mst.edu/civeng110/concepts/06/concentration/index.html#:~:text=The%20ratio%20of%20the%20true,of%20stress%20raiser%20in%20question.}{Torsion Stress Concentrations in Circular Shafts}
	\end{itemize}


	\subsection{Inelastic Torsion}
	\begin{itemize}
		\item If the applied torque is more than a certain threshold (yielding torque), the hooke's law ($\tau = G \gamma$) is no longer valid. But the strain still varies linearly across the cross section i.e, $\gamma = r\dv{\phi}{x}$ since in the derivation of this(\ref{torsion}), we only used basic geometry.

		\item This is valid for inelastic torsion
		\[T = \int \rho \tau dA = 2 \pi \int_0^r \tau \rho^2 d\rho\]


	\end{itemize}

	\subsection{Elasto plastic Torque}
	\begin{itemize}
		\item Consider a bar with circular cross section made up of elastic plastic material with torque applied along its axis.

		\item If we keep increasing torque, at yielding torque ($T_Y$), the outer boundary of the bar yields while the rest of the circular cross section is still in the linear region. As we keep increasing torque, the material yield further toward the center.

		\item At limiting torque $T_P$, the bar completely yields.

		\item The shear stress in the yielded part of the bar is constant and equal to the yielding shear stress $\tau_Y$.

		\item We can calculate T in terms of $\tau$ at any point after yielding. (Here, $\rho_Y$ is the radius of the material which has not yielded)
		\begin{align*}
			T &= \int \rho \tau dA\\
			  &= 2\pi \int_0^r \tau \rho^2 d\rho\\
			  &= 2\pi \left(\int_0^{\rho_Y} (\frac{\tau_Y}{\rho_Y} \rho) \rho^2 d\rho + \int_{\rho_Y}^r \tau_Y \rho^2 d\rho \right)\\
			  &= \frac{\pi \tau_Y}{6} (4r^3 - \rho^3_Y)
		\end{align*}
		\[\boxed{T=\frac{\pi \tau_Y}{6} (4r^3 - \rho^3_Y)}\]

		\item After the material has completed yielded $\rho_Y = r$, substituting in above equation
			\[T_P = \frac{2\pi}{3}\tau_Y r^3 = \frac{4}{3}T_Y\]
			\[\boxed{T_P = \frac{4}{3} T_Y}\]

		\item Ref: \href{https://classes.mst.edu/civeng2211/lessons/torsion/theory_full/index.html}{Torsion Loading}

	\end{itemize}


	\subsection{Power due to torque}
	\begin{itemize}
		\item Power due to applied torque,
		\[P = \dv{W}{t} = \frac{Td\theta}{dt} = T\omega\]

	\end{itemize}


\section{Bending}

	\subsection{Sign Convention}
		Sign convention that prof is following in this course: On rightward end, anticlockwise moment and downward shear force is taken as positive


	\subsection{Relation between V, M and q}
		Consider a rod under stress. Let the distributed load on it be q(x) (upwards is +ve), shear force V(x) and bending moment M(x). Consider a small element of length $dx$ at a distance x. \\

		\noindent \textbf{Shear force}
		\begin{gather*}
			\sum F_y = 0\\
			V + q dx - (V+dV) =0 \\ %\qquad(\text{distributed load can be considered constant})
			\boxed{\dv{V}{x} = - q}
		\end{gather*}

		\noindent \textbf{Bending Moment}
		\begin{gather*}
			\sum M = 0 \\
			(M + dM) - M + qdx (\frac{dx}{2}) - Vdx =0\\
			\dv{M}{x} = V \text{      (since dx dx term = 0 )}
		\end{gather*}

		\[\boxed{\dv{M}{x} = V}\]


	\subsection{Pure bending}
		Ref: \href{https://www.youtube.com/watch?v=f08Y39UiC-o}{Youtube video}\\
		Let a beam be under bending moment having a radius of curvature $\rho$. Then curvature $\kappa$ is defined as
		\[\kappa = \frac{1}{\rho} = \dv{\theta}{s}\]
		(since $\rho = \dv{s}{\theta}$)\\

		Curved upward is the positive curvature.\\

		For a small length element $dx$ from distance of length y (upwards) from the neutral axis and $d\theta$ be the angle it subtends at the center of curva (take x axis along the neutral axis). Then the strain in that element is
		\begin{align*}
			\sigma_x &= E \epsilon_x \qquad \text{(Assuming Hooke's law)}\\
					 &= E \frac{(\rho -y)d\theta - dx}{dx}\\
					 &= E\frac{(dx - yd\theta) - dx}{dx}\\
					 &= -\frac{Ey}{\rho}\\
					 &= -E\kappa y
		\end{align*}	
		\[\boxed{\sigma_x = - \frac{Ey}\rho} \quad \text{(Assuming Hooke's law)}\]	


		Since, the beam is in equilibrium, (in the following calculation, the area (vector) is along x direction - since we want to sum $\sigma_x$)
		\begin{align*}
			\int_A \sigma_x dA &= 0\\
			\implies \int_A -\frac{Ey}{\rho}dA &=0\\
			\implies \int_A ydA &=0
		\end{align*}
		This implies that the neutral axis should pass through the centroid. (or $\int_A ydA = 0$) 

	\subsection{Flexure Formula}
		Continuing the above case, the sum of all the moments is equal to the bending moment.
		\begin{align*}
			M &= - \int_A \sigma_xy dA \qquad \text{(-ve sign because +ve $\sigma$ produces a negative moment)}\\
			  &= - \int_A \frac{-Ey}{\rho} y dA\\
			  &= \frac{E}{\rho} I 
		\end{align*}
		where $I = \int_A y^2 dA$ is the second moment of area.

		% \[\boxed{\kappa = \frac{1}{\rho} = \frac{M}{EI}} \]
		\[ {M} = \frac{EI}{\rho} \]
		$EI$ is known as the \textbf{Flexure Rigidity}
		\[\boxed{\sigma_x = \frac{-My}{I}}\]

		\textbf{Maximum Bending Moment}: It occurs at maximum distance (y) from the neutral axis.\\

		Let $c_1$ and $c_2$ be maximum distances (along y) from the neutral axis in either directions, then \textbf{section modulus} is 
		\[S_1 = \frac{I}{c_1}\]

		Maximum longitudinal (along x) is
		\[\boxed{(\sigma_x)_{max} = \frac{M_{max}c}{I}}\]

\section{Transverse Shear Stress}
		Consider a small element (along lenght) of a rod under stress. Let the bending on left end be M and right be M+dM and shear force V and $V+dV$. Consider the top portion of the element upto y distance from the neutral axis (top to y). Let $\sigma_1$ and $\sigma_2$ be the variation of logitudinal stress.Then, 
		\begin{align*}
			\tau dA \text{(along x-z plane}) &= \int (\sigma_2 - \sigma_1) dA \text{(along y-z plane)}\\
											 &= \int (\frac{(M+dM)y}{I} - \frac{My}{I} dA  )\\
								\text{Assuming that transverse stress is} & \text{ uniformly distributed along the width}\\
									\tau bdx &= \frac{dM}{I} \int ydA \\
										\tau &= \frac{dM}{dx} \frac{1}{Ib} \int ydA\\
		\end{align*}
		\[\boxed{\tau = \frac{vQ}{Ib}}\]
		where, $Q=\int ydA$, area varies from the top portion to the untill plane we want to compute $\tau$. This is called \textbf{shear formula}.\\

		Ref : \href{https://www.youtube.com/watch?v=4x0E9yvzfCM}{Youtube video}\\


		For rectangualar cross sections, Consider a portion above the height $y_1$ above the neutral axis,
		\[Q = \int_{y_1}^{\frac{h}{2}} ybdy = \frac{b}{2}\left(\frac{h^2}{4}- y^2 \right)\]
		\[\tau= \frac{VQ}{Ib} = \frac{Vb}{2I}\left(\frac{h^2}{4}- y^2 \right)\]
		At $y=0$,
		\[\tau_{max} = \frac{3}{2} \frac{V}{A}\]

		% (//todo:limitations)

		For circlular cross sections at the center, 
		\[Q = \frac{\pi r^2}{2}\frac{4r}{3\pi} = \frac{2r^3}{3}\]
		\[\tau_{max} = \frac{VQ}{Ib} = \frac{V(\frac{2}{3}r^3)}{(\frac{\pi r^4}{4})(2r)} = \frac{4V}{3A}\]

		For hollow cyclinder, 
		\[\tau_{max} = \frac{V (\frac{2}{3}(r_2^3 -r_1^3))}{(\frac{\pi}{4}(r_2^4 - r_1^4))(2(r_2-r_1))}= \frac{3V}{4A} \left( \frac{r_2^2 + r_1 r_2 + r_1^2}{r_2^2+r_1^2} \right)\]


\pagebreak

\section{Neutral Axis}
	\begin{itemize}
		\item For moments along y and z with resultant moment making an angle of $\theta$ with the z axis then the angle made by neutral axis with the z axis $\beta$ is given by
		\[ \tan \beta = \frac{I_z}{I_y} tan\theta\]




	\end{itemize}

\section{Shear Center}
	\begin{itemize}
		\item If the line of action of any shear force is passes through the shear center, then it does not cause any twisting. 
		\item This point is unique.
		\item It need not always lie inside the material.
		\item Shear stress for a load applied along y axis for any open section cut with thin cross section from a distance s from the free edge is 
		\[\tau = \frac{V_yQ_y}{I_zt}\]
		where $Q_y = \int_0^s y dA$ is the first moment of the area computed from the free end to the point along the material wrt neutral axis

		\item Shear flow can be thought of as a fluid, if the shear flow from 2 branches join, then the new shear flow is sum of the two shear flows.
		\item To calculate shear force along a direction, we should calculate first moment of area using the distance from the axis perpendicular to that axis.


	\end{itemize}

\section{Combined Loading}
	\subsection{Spherical vessel}
	\begin{itemize}
		\item Suppose that there is uniform pressure p inside a hollow spherical vessel of radius r and thickness t. Then cutting the sphere along any cross section and equating the forces to zero. Then the normal stress in the hollow spherical vessel is
		\[\sigma = \frac{pr}{2t}\]
		\item (In plane shear stress) This normal stress $\sigma$ is uniform in all directions and hence, $\tau_{xy} = 0$
		\item (Out of plane shear stress) If we consider z direction along with x/y direction, then the maximum shear stress can be found out using mohr circle as 
		\[\tau_{xz} = \frac{pr}{4t}\]

		\item These conclusions were by assuming the ratio $\frac{r}{t} >> 1$. (or else $\sigma_z = -p$ would affect the result)
	\end{itemize}


	\subsection{Cylindrical vessel}
	\begin{itemize}
		\item For a cylindrical vessel, there are two different tensile stresses, i.e, along the curved surface and along the height
		\item (Circumferential stress) Making a cut along the height to calculate the tensile stress along the curved surface, we have
			\[\sigma_1 = \frac{pr}{t}\]

		\item (Longitudinal stress) Similarly cut along the radial direction to calculate tensile stress along the height gives us
			\[\sigma_2 = \frac{pr}{2t}\]
			\[\sigma_1 = 2 \sigma_2\]

		\item (In plane shear stress) The maximum shear stress calculated along in the x-y plane (x-y plane has the small cylindrical part) is not zero and can be calculated using mohr's circle
			\[\tau_{xy} = \frac{\sigma_1 - \sigma_2}{2} = \frac{pr}{4t}\]

		\item (Out of plane shear stress) The maximum shear stress considering the x-z (z is perpendicular to the small cylindrical part) is 
			\[\tau_{xy} = \frac{\sigma}{2} = \frac{pr}{2t}\]

	\end{itemize}

\section{Deflection}

	\begin{itemize}
		\item For a linearly elastic following Hooke's law beam under with moment of inertia I, radius of curvature $\rho = \frac{1}{\kappa}$ under a moment M, the beam gets deflected by a deflection $v$, then
			\[\dv[2]{v}{x} = \kappa = \frac{M}{EI}\]
		\item For small deflections, these equations hold good
			\[EI v'' = M \qquad EI v''' = V \qquad EI v'''' = -q\]

		\item For large deflections, 
			\[\kappa = \frac{v''}{(1+v'^2)^{\frac{3}{2}}}\]

		\item Calculation of Strain Energy using bending moment - 
			\[ U = \int \frac{M^2 dx}{2EI}\]

		\item Castigliano's first theorem
			\[ P = \dv{U}{\delta} \]
		\item Castigliano's second theorem
			\[\delta = \dv{U}{P} \]

		\item Modified costigliano's theorem - In this instead of integrating for energy then differentiating, we first differentiate and then integrate
			\[ \delta = \int \frac{M}{EI}\pdv{M}{P} dx\]

		\item Temperature Effect
			\[\dv{\theta}{x} = \dv[2]{v}{x} = \frac{\alpha (T_2 - T_1)}{h}\]

		\item Longitudinal displacement effect
			\[ \lambda = \frac{1}{2} \int_0^L (\dv{v}{x})^2 dx\]

	\end{itemize}




\section{Buckling}
	\subsection{Uniform beam in compression}
	\begin{itemize}
		\item For buckling to happen, the force P has to be more than $P_0$, 
				\[\sqrt{\frac{P}{EI}} L = n \pi \]

		\item For buckling to occur, critical stresses
				\[P = \frac{n^2 \pi^2 EI}{L^2}\]

		\item When unconstained at all points, the buckling will take place at $n=1$, or
				\[P_0 = \frac{\pi^2 EI}{L^2}\]

		\item Critical stress
			\[\sigma_{cr} = \frac{\pi^2 EI}{A L^2} = \frac{\pi^2 E}{(L/r)^2}\]

		where radius of gyration $r = \frac{I}{A}$,\\
		and Slenderness ratio = $\frac{L}{r}$
	\end{itemize}


\section{Second Moment of Inertias}
	\begin{itemize}
		\item For circle of radius $r$
			\[I_x = \frac{\pi}{4} r^4\]
			\[I_z = \frac{\pi}{2} r^4\]
		\item For circle of hollow circle radius $r_1$ and $r_2$
			\[I_x = \frac{\pi}{4} (r_1^4 - r_2^4)\]
			\[I_z = \frac{\pi}{2} (r_1^4 - r_2^4)\]

		\item For a rectangle with base b and height h
			\[I_x = \frac{bh^3}{12}\]
			\[I_y = \frac{b^3h}{12}\]

	\end{itemize}


























\end{document}
