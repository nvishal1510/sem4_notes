\documentclass{article}
\title{ME 213L Manufacturing processes Lab}
\author{Vishal Neeli}

\usepackage[a4paper, total={6in, 11in}]{geometry}
\usepackage{textcomp}
\usepackage{hyperref}
\usepackage{amsmath}
\usepackage{xcolor}
\usepackage{grffile}
\usepackage{physics}
\usepackage{amsfonts}
\usepackage{amsthm}
\usepackage{array}
\hypersetup{
	colorlinks=true,
	urlcolor=blue,
	linkcolor=cyan,
	filecolor=red
}
\newtheorem*{theorem}{Theorem}

% FOR CODE
% \usepackage{listings}
% \usepackage{color}

% \definecolor{dkgreen}{rgb}{0,0.6,0}
% \definecolor{gray}{rgb}{0.5,0.5,0.5}
% \definecolor{mauve}{rgb}{0.58,0,0.82}

% \lstset{frame=tb,
%   language=Java,
%   aboveskip=3mm,
%   belowskip=3mm,
%   showstringspaces=false,
%   columns=flexible,
%   basicstyle={\small\ttfamily},
%   numbers=none,
%   numberstyle=\tiny\color{gray},
%   keywordstyle=\color{blue},
%   commentstyle=\color{dkgreen},
%   stringstyle=\color{mauve},
%   breaklines=true,
%   breakatwhitespace=true,
%   tabsize=3
% }

\begin{document}
\maketitle

\section{Machine codes}
\begin{tabular}{|m{12em}|m{30em}|}
	\hline
	\textbf{Syntax} & \textbf{Function}\\
	\hline
	\hline
	G54 & workpiece coordinate system\\
	\hline
	G90 & absolute coordinate system\\
	\hline
	G91 & moving coordinate system(attached to tool)\\
	\hline
	G00 X$x_f$ Z$z_f$ & The tool goes from current position to $x_f$ and $z_f$\\
	\hline
	G01 X$x_f$ Z$z_f$ Ff & tool moves from current position to $x_f$ and $z_f$ at feed rate f.\\
	\hline
	G02 X$x_f$ Z$z_f$ R$r$ Ff & Circular interpolation - the tool moves to $x_f,z_f$ along the arc of radius $r$ at feed rate $f$ in clockwise sense\\
	\hline
	G03 X$x_f$ Z$z_f$ R$r$ Ff & Circular interpolation - the tool moves to $x_f,z_f$ along the arc of radius $r$ at feed rate $f$ in anticlockwise sense\\
	\hline
	G32 X$x_{minor}$ Z$z_f$ F$p$ & Threading - Execute this command from $X=x_{minor}$ and Z outside the work piece. Here, $z_f$ is the min diameter point of the last thread\\
	\hline
	G73 U$u$ W$w$ R$r$ & Canned cycle - u and z are the radial distance and z distance of the cut, r is the number of cyles\\
	G73 P$p$ Q$q$ U$u$ W$w$ F$f$ S$s$ & Canned cylce - p and q are the start and end line numbers of the path. u and w are the radial and z direction finishing allowances.\\
	\hline
	G28 U00 V00 & Moves to home \\%(// todo: addition req)
	\hline
	M03 Ss & Spindle rotates clockwise at s rpm\\
	\hline
	M04 Ss & Spindle rotates anticlockwise at s rpm\\
	\hline
	M05 & Spindle stops\\
	\hline
	M06 Tt & Tool changes to tool t\\
	\hline
	M30 & Program ends\\
	\hline
	
\end{tabular}


\section{Tools to use}

\begin{itemize}
	\item Facing: It is the reduction of length of workpiece (along z) - Use tool 5
	\item Turning: Reducing the diameter - use tool 5
	\item Grooving: Make a cavity into the workpiece - use tool 98
	\item Circular interpolation: use tool 5
	\item Threading: use tool 103
	\item Drilling: use tool 174
	\item Boring: use tool 145
	\item Internal grooving: use tool 168
	\item Internal threading: use tool 171

\end{itemize}












\end{document}