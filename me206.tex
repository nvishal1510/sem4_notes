\documentclass{article}
\title{}
\author{Vishal Neeli}

\usepackage[a4paper, total={6in, 11in}]{geometry}
\usepackage{textcomp}
\usepackage{hyperref}
\usepackage{amsmath}
\usepackage{xcolor}
\usepackage{grffile}
\usepackage{physics}
\hypersetup{
	colorlinks=true,
	urlcolor=blue,
	linkcolor=cyan,
	filecolor=red
}
\usepackage{amsfonts}

% FOR CODE
% \usepackage{listings}
% \usepackage{color}

% \definecolor{dkgreen}{rgb}{0,0.6,0}
% \definecolor{gray}{rgb}{0.5,0.5,0.5}
% \definecolor{mauve}{rgb}{0.58,0,0.82}

% \lstset{frame=tb,
%   language=Java,
%   aboveskip=3mm,
%   belowskip=3mm,
%   showstringspaces=false,
%   columns=flexible,
%   basicstyle={\small\ttfamily},
%   numbers=none,
%   numberstyle=\tiny\color{gray},
%   keywordstyle=\color{blue},
%   commentstyle=\color{dkgreen},
%   stringstyle=\color{mauve},
%   breaklines=true,
%   breakatwhitespace=true,
%   tabsize=3
% }

\begin{document}
\maketitle

\section{Introduction}

\begin{itemize}
	\item The five M's of the manufacturing define the profitability of the manufacturing process
	\begin{itemize}
		\item Men (women)
		\item Machine
		\item Materials
		\item Money 
		\item Management

	\end{itemize}

	\item Manufacturing processes can be classified into:
	\begin{itemize}
		\item Shaping processes - alter the shape
		\begin{itemize}
			\item Solidification process
			\item Metal Removal process
			\item Deformation process
			\item Assembly process
		\end{itemize}
		\item Property enhancing processes - change the properties of the material\\
		Ex: annealing, hardening
		\item Surface enhancing processes - clean the surfaces or coat them\\
		Ex: Surface oxidation
	\end{itemize}


	\item Selection of manufacturing processes depends on various factors
	\begin{itemize}
		\item Cost 
		\item Available infrastructure
		\item Operating temperature
		\item Required properties
		\item Dimensions of required product
	\end{itemize}

\end{itemize}

\section{Casting Process}

Pros:
\begin{itemize}
	\item Complex forms, low cost
	\item Certain shapes cannot be machined
	\item One piece parts vs. multiple piece parts
	\item Design changes are easily incorporated
	\item High volume, low skilled labor
	\item Large, heavy parts can be made easily
\end{itemize}

Cons:
\begin{itemize}
	\item Problems with internal porosity
	\item Dimensional variations due to shrinkage
	\item Trapped impurities, solids and gasses
	\item High-tolerance, smooth surfaces not possible
	\item More costly than stamping or extruding in some cases
\end{itemize}

\subsection{Pattern} 
\begin{itemize}
	\item It is the replica (usually a bit bigger) of the req product 
	\item It should be cost efficient, easily machinable into common shapes.\\
	Possible choices are: 
	\begin{itemize}
		\item Wood : more than 90\% of the production happens with wood, absorbs moisture, relatively lower life
		\item Metal: longer life, used in large quantity production, Al, cast iron and bronze are most used
		\item Plastic : good corrosion resistance, smooth surfaces, do not absorb moisture, dimensionally stable, low weight
	\end{itemize}
	\item Pattern types
	\begin{itemize}
		\item Single piece pattern: Name says it all
		\item Split pattern
		\item Cope - split pattern: Same as split pattern except that parts are moulded seperately
		\item Gated pattern: Tree like structure with each branch holding the required shape
	\end{itemize}
\end{itemize}

\subsection{Allowances}
\begin{itemize}
	\item Draft/ Taper Allowance\\
		- To avoid damage to the sand mould on the internal walls, we need to provide extra material on the walls of the pattern. We typically use a tapering angle(angle at which pattern is bent outwards).

	\item Machining/Finishing allowance\\
		- It is the extra material allowed to cut to smooth finish the product. The amount of machining depends on method of moulding method of molding and casting used, size and shape of casting, metal used in casting, required accuracy and finish

	\item Distortion / Camber allowance\\
	 	- Due to typical shape of some patterns (non symmetric) (T, U, V, W), after cooling down the shape will change due to uneven shrinkage, so we need to allow distortion allowance i.e, we need to bend the pattern in opp dir to the expected distortion to get correct shape. This varies between 2 to 20 mm.
\end{itemize}

\subsection{Molds}
\begin{itemize}
	\item Expendable Molds: A new mold has to be produced for every cycle. Types: green sand, dry sand, shell, investment, plaster
	\item Permanent molds: Multiple usages, used in: die casting, centrifugal, pressure die, injection molding

	\item Required properties of moulding sand:
	\begin{itemize}
		\item Refractoriness: Ability to withstand higher temperatures
		\item Permeability: For the air to escape
		\item Cohesiveness: Adhesive forces in sand
		\item Flowability: It should flow uniformly to all the portions
	\end{itemize}

\end{itemize}


- A top gating design is prefered over bottom gating design and parting gating design because in the case of top gating design, the gravity help the molten metal to flow to intricate corners of the pattern.





\subsection{Types of Casting}

\begin{itemize}
\item Sand casting:\\
	Advantages
	\begin{itemize}
		\item This is cheap
		\item This can be used ferrous and non-ferrous materials. Non ferrorus metals require high temp
	\end{itemize}

	Disadvantages
	\begin{itemize}
		\item Rough finish
	\end{itemize}
	\item

\item Investment casting
	\begin{itemize}
		\item We first create a pattern out of wax(optimisied for this process)  and then dip it in slurry (silica). The slurry solidify on the outer surface to req thickness. Then we heat the thing to melt wax and remove the wax completely to create a outer wall for the req shape. The liq metal is poured into the solid slurry to create the req shape. After cooling down, outer shell can be removed
		\item This has a smooth wall finish
		\item It can be used to create complex shapes because wax can be easily bended 
		\item Also known as lost wax casting precision casting
		\item This is relatively expensive compared to sand casting
	\end{itemize}


\item Centrifugal casting
	\begin{itemize}
	\item Molten metal is poured into a long cylinder and rotated, and it is cooled on the outer surface
	\item The outer part cools faster and have smaller grain size
	\item The lighter impurities present in the metal get accumulated at the inner surface. This can be removed by machining process in the inner surface
	\end{itemize}

	
\item Shell moulding
	\begin{itemize}
	\item The req shape in metal in heated and some silica or sand made to stick to the metal and a outer shell is made.
	\item Then liquid metal is poured into the mould.
	\item Axial symmetry things can be made easily by this.
	\end{itemize}


\item Lost Foam/ Evaporation Pattern
	\begin{itemize}
	\item A pattern is created is using polystryene and sand is used to support it.
	\item The metal is directly poured into the polystryene and the polymer evaporates and creating req shape
	\end{itemize}


\item Continuous casting
	\begin{itemize}
	\item Partially solidified (outer surface) metal is sent through the roller and cooling liquid is sprayed
	\item This creates uninterupted long strands of metal

	\end{itemize}



 \item (Pressure) Die casting
 	\begin{itemize}
 	\item High cost
 	\item The dies are two shapes which lock into each creating a cavity in between. 
 	\item Before locking they are sprayed with some liq to prevent sticking of molten metal to die
 	\item Metal is poured is cavity and pressure is applied to molten metal. Pressure ensures that metals reaches all the cavities and ensures that metal reaches everywhere before it start solifidies. After removing ejector pins, dice get seperated.
 	\item Cooling channels are present in the die which cool the metal. The whole can be completed as fast as 1min.
 	\item In cold chamber die casting - higher melting point metals (prefered because it can cost higher to maintain metal in liq state) and in hot cold chamber die casting - lower melting metal (can be used for higher melting metal)
	\end{itemize}


\end{itemize}


\subsection{Casting Defects}
	\begin{itemize}
 	\item Misrun : metal solidified before reaching all places
 	\item cold shut : meeting of two metal streams
 	\item cold shot : resulted due to turbulence during casting process
	\end{itemize}
 	

\end{document}